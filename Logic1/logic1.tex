\documentclass[10pt]{article}
\usepackage[utf8]{inputenc}
\usepackage[T1]{fontenc}
\usepackage{amsmath}
\usepackage{amsfonts}
\usepackage{amssymb}
\usepackage{mhchem}
\usepackage{stmaryrd}

\title{Problem Set 1 - second try Advanced Logic, Fall 2020 }

\author{}
\date{}


\begin{document}
\maketitle
As I explained, I am making this available to allow for the possibility of a "do-over" if you are not satisfied with the answers you turned in for Problem Set 1 on Friday.

If you wish, you may turn solutions any of the following problems at any time before class on Tuesday September 13th. If you turn in a solution, it will replace whatever you may have turned in the first time around for the correspondingly numbered problem.

If you're turning in answers for this problem set, please help us keep track, by listing any cases where you want us to \textit{keep} what you turned in first time around, and if possible, copy and paste the answer you want us to keep into the document containing your answers to these new problems.

\begin{enumerate}
  \item (50\%) Prove that for any sets $A$ and B, $A \subseteq B$ if and only if $A \backslash B=\varnothing$.
    \begin{enumerate}
      \item Assume that $A \subseteq B$ and show that $A \backslash B=\varnothing$.
      \begin{enumerate}
        \item Since $A \subseteq B$, for any element $a$ in A, $a \in A$, $a \in B$.
        \item By the definition of $\backslash$, $a \in A \backslash B$ if and only if $a \in A$ and $a \notin B$.
        \item Therefore, if $a \in A \backslash B$, then $a \in A$ and $a \notin B$. 
        \item Combining (i) and (iii), we reach that $a \in B$ and $a \notin B$, thus $ a \neq a$.(contradiction)
        \item By the definition of $\varnothing$, $\varnothing:=\{x \mid x \neq x\}$
        \item and substitue x for $a$ by(iv), $a \neq a$, therefore, we satisfied the definition of empty set
        \item for every element in $A \backslash B$, $a \in A \backslash B$, $a \in A$, $a \notin B$, $a \neq a$ , $a \in \varnothing $ 
        therefore, $A \backslash B \subseteq \varnothing$
        \item By definition of $\varnothing$, $\varnothing$ is the subset of anyset, Thus, $\varnothing \subseteq A \backslash B$
        \item Therefore, $A \backslash B = \varnothing$
      \end{enumerate}
      \item Assume that $A \backslash B=\varnothing$ and show that $A \subseteq B$.
      \begin{enumerate}
        \item To prove that A is a subset of B, we need to show that for any element $a$ in A, $a \in B$.
        \item With the assumption that $A \backslash B=\varnothing$, for any element $a$ in $A \backslash B$, $ a \in A$, $a \notin B$ and $a \in \varnothing$
        \item Since $a$ represents any element in A, $a \in A$ and $a \in \varnothing$, $A \subseteq \varnothing$
        \item Since $\varnothing$ is the subset of any set, $\varnothing \subseteq A$
        \item Combining (iii) and (iv), we reach that $A = \varnothing$
        \item Thus, $A \subseteq B$ (empty set is the subset of any set)
      \end{enumerate}
    \end{enumerate}
  \item (30\%) Prove that whenever $R$ is a relation from $A$ to $B$ and $S$ is a relation from $B$ to $C$,
  \begin{enumerate}
    \item If $R$ and $\mathrm{S}$ are both surjective, then $S \circ R$ is surjective.
      \begin{enumerate}
        \item By the definition of surjective, for every element in set C there is some element in set B such that $\langle b, c\rangle \in S$.
        \item And for every element in set B there is some element in set A such that $\langle a, b\rangle \in R$.
        \item Combining(i) and (ii), for every element in set C there is some element in set B and for every element in set B there is some element in A
        \item By the definition of composition($S \circ R$), $\{\langle x, z\rangle \in A \times C \mid$ there exists $y \in B$ such that $\langle x, y\rangle \in R$ and $\langle y, z\rangle \in S\}$, from $A$ to $C$
        \item Substitue x, y, z with a, b, c, respectively, we reach that $\{\langle a, c\rangle \in A \times C \mid$ there exists $b \in B$ such that $\langle a, b\rangle \in R$ and $\langle b, c\rangle \in S\}$
        \item By the definition of $\phi$(generic act), Axiom of Product Existence(I dont know how far I should go for the existence proof, it's a big rabbit hole), there exists a, b, c.
        \item So, $S \circ R$ is surjective.
      \end{enumerate}
    \item $R$ and $S$ are both injective, then $S \circ R$ is injective.
      \begin{enumerate}
        \item By the definition of injective, $R$ is injective iff whenever $R x y$ and $R x^{\prime} y, x=x^{\prime}$.
        \item By the given condition that R is injective, $R a b$ and $R a^{\prime} b, a=a^{\prime}$.
        \item And $S$ is injective, $S b c$ and $S b^{\prime} c, b=b^{\prime}$.
        \item By the definition of composition($S \circ R$), $\{\langle x, z\rangle \in A \times C \mid$ there exists $y \in B$ such that $\langle x, y\rangle \in R$ and $\langle y, z\rangle \in S\}$, from $A$ to $C$
        \item Substitue x, y, z with a, b, c, respectively, we reach that $\{\langle a, c\rangle \in A \times C \mid$ there exists $b \in B$ such that $\langle a, b\rangle \in R$ and $\langle b, c\rangle \in S\}$
        \item Similarly to (a), we reach that $S \circ R$ is injective.
      \end{enumerate}
  \end{enumerate}
  \end{enumerate}
\begin{enumerate}
  \setcounter{enumi}{3}
  \item (10\%) Let $\mathrm{A}=\{a, b\}$ be a 2-membered set $\mathrm{B}=\{c, d, e\}$ be a three-membered set. List all the functions from A to B (there are $9=3^{2}$ ) and all the functions from $B$ to $A$ (there are 8 $=2^{3}$ ). For each of these 17 functions, specify whether it is injective and whether it is surjective.
  \begin{enumerate}
    \item $\mathrm{A}=\{a, b\}$ be a 2-membered set $\mathrm{B}=\{c, d, e\}$ be a three-membered set.
    \item List all the functions from A to B (there are $9=3^{2}$ )
    \begin{enumerate}
      \item With the condition: $f(a)=c$
      \item $\mathrm{f}_{1}=\{\langle a, c\rangle, \langle b, d\rangle\}$, Injective
      \item $\mathrm{f}_{2}=\{\langle a, c\rangle, \langle b, e\rangle\}$, Injective
      \item $\mathrm{f}_{3}=\{\langle a, c\rangle, \langle b, c\rangle\}$, Surjective
      \item With the condition: $f(a)=d$
      \item $\mathrm{f}_{4}=\{\langle a, d\rangle, \langle b, c\rangle\}$, Injective
      \item $\mathrm{f}_{5}=\{\langle a, d\rangle, \langle b, d\rangle\}$, Surjective
      \item $\mathrm{f}_{6}=\{\langle a, d\rangle, \langle b, e\rangle\}$, Injective
      \item With the condition: $f(a)=e$
      \item $\mathrm{f}_{7}=\{\langle a, e\rangle, \langle b, c\rangle\}$, Injective
      \item $\mathrm{f}_{8}=\{\langle a, e\rangle, \langle b, d\rangle\}$, Injective
      \item $\mathrm{f}_{9}=\{\langle a, e\rangle, \langle b, e\rangle\}$, Surjective
    \end{enumerate}
    \item all the functions from $B$ to $A$ (there are 8 $=2^{3}$ )
    \begin{enumerate}
      \item With the condition: $f(c)=a$
      \item $\mathrm{g}_{1}=\{\langle c, a\rangle, \langle d, a\rangle\}$, Surjective
      \item $\mathrm{g}_{2}=\{\langle c, a\rangle, \langle d, b\rangle\}$, Injective
      \item $\mathrm{g}_{3}=\{\langle c, a\rangle, \langle e, a\rangle\}$, Surjective
      \item $\mathrm{g}_{4}=\{\langle c, a\rangle, \langle e, b\rangle\}$, Injective
      \item With the condition: $f(c)=b$
      \item $\mathrm{g}_{5}=\{\langle c, b\rangle, \langle d, a\rangle\}$, Injective
      \item $\mathrm{g}_{6}=\{\langle c, b\rangle, \langle d, b\rangle\}$, Surjective
      \item $\mathrm{g}_{7}=\{\langle c, b\rangle, \langle e, a\rangle\}$, Injective
      \item $\mathrm{g}_{8}=\{\langle c, b\rangle, \langle e, b\rangle\}$, Surjective

    \end{enumerate}
  \end{enumerate}
  \item (10\%) Prove that where $R$ is a relation from $A$ to $B$ and $S$ and $T$ are relations from $B$ to $C,(S \cup T) \circ R=(S \circ R) \cup(T \circ R) \cdot$
  \begin{enumerate}
    \item To prove that $(S \cup T) \circ R=(S \circ R) \cup(T \circ R)$, we need to prove that 
    \item Left Side: $(S \cup T) \circ R$ is a subset of $(S \circ R) \cup(T \circ R)$
      \begin{enumerate}
        \item Since binary relations can be defined as a set of ordered pairs which is a subset of Cartesian Product.
        \item It is a special case of set, thus, we can use set operations on it.
        \item $S$ and $T$ are some relations from $B$ to $C$. Therefore, the Union of $S$ and $T$ is also a relation from $B$ to $C$, we can call it K.
        \item $K \circ R $ means that the composition of R and K is the relation that $\{\langle x, z\rangle \in A \times C \mid$ there exists $y \in B$ such that $\langle x, y\rangle \in R$ and $\langle y, z\rangle \in K\}$, from $A$ to $C$
        \item So, every ordered pair $\langle x, z\rangle$ in $K \circ R$ is a subset of $(S \cup T) \circ R$
        \item Next, I am gonna prove that every ordered pair $\langle x, z\rangle \subseteq (S \cup T) \circ R$ is a subset of $(S \circ R) \cup(T \circ R)$
        \item By the definition of composition, $S \circ R$ is the composition of R and S, which is the relation that $\{\langle x, z\rangle \in A \times C \mid$ there exists $y \in B$ such that $\langle x, y\rangle \in R$ and $\langle y, z\rangle \in S\}$, from $A$ to $C$
        \item Thus, every ordered pair $\langle x, z\rangle \subseteq S \circ R$
        \item Similarly, $\langle x, z\rangle \subseteq T \circ R$
        \item Thus, (vi) is proved. 
        \item Combining (vi) and (v), we can conclude that $(S \cup T) \circ R$ is a subset of $(S \circ R) \cup(T \circ R)$
      \end{enumerate}
    \item Right Side: $(S \circ R) \cup(T \circ R)$ is a subset of $(S \cup T) \circ R$ 
    \begin{enumerate}
      \item Using the same method as the left side, but assuming that $(S \circ R) \cup(T \circ R)$, the ordered pair $\langle x, z\rangle$ is a subset of $(S \cup T) \circ R$
    \end{enumerate}
  \end{enumerate}
  
  \item (10\%) Show that where $R$ is a relation from $A$ to $B$,
  \begin{enumerate}
    \item $R$ is surjective iff id $\mathrm{B} \subseteq R_{\circ} R^{-1}$
    \begin{enumerate}
      \item From left to right
      \begin{enumerate}
        \item Assume that R is a relation from A to B, it is surjective iff for every $b \in B$, there exists some $a \in A$ such that $\langle a, b\rangle \in R$
        \item By the definition of converse, $R^{-1}$ is the relation that $\{\langle b, a\rangle \mid \langle a, b\rangle \in R\}$, from $B$ to $A$
        \item Since $R$ is surjective, for every $b \in B$, there exists some $a \in A$ such that $\langle a, b\rangle \in R$
        \item According to the assumption of set B exists, and the definition of $\operatorname{id}_{B}$, for any set, $\langle b, b\rangle \in \operatorname{id}_{B}$
        \item Thus, besides, R is surjective from A to B, R is also surjective from R to R.
        \item Since the composition of a relation and its converse is the identity relation, $R_{\circ} R^{-1}$ is the identity relation, which is the same as $\operatorname{id}_{B}$
        \item Thus, $\operatorname{id}_{B} \subseteq R_{\circ} R^{-1}$
      \end{enumerate}
    \item From right to left
      \begin{enumerate}
        \item From right to left
        \item It is assumed that B exists, therefore $\operatorname{id}_{B}$ also exists.
        \item It is also assumed that R exists, thus, $R_{\circ} R^{-1}$ is just R itself by definition of composition
        \item According to the PowerPoint(a few noteworthy fact), the composition of $\operatorname{id}_{B}$ and R is just R itself. 
        \item thus, $\operatorname{id}_{B} \subseteq R$
        \item thus, for every $b \in B$, there exists some $a \in A$ such that $\langle a, b\rangle \in R$ as R is defined by a relation from A to B
        \item As a result, R is surjective.
      \end{enumerate}
    \end{enumerate}
    \item $R$ is functional iff $R_{\circ} R^{-1} \subseteq \mathrm{id}_{\mathrm{B}}$
  \end{enumerate}
\end{enumerate}
\end{document}