\documentclass[10pt]{article}
\usepackage[utf8]{inputenc}
\usepackage[T1]{fontenc}
\usepackage{amsmath}
\usepackage{amsfonts}
\usepackage{amssymb}
\usepackage{mhchem}
\usepackage{stmaryrd}
\usepackage{mathrsfs}

\title{Advanced Logic: Problem Set 02 }

\author{}
\date{}


\begin{document}
\maketitle
\begin{itemize}
  \item Due date: Friday, September $16 .$

  \item Your proofs may appeal to any facts stated in the first three lectures.

\end{itemize}
\begin{enumerate}
  \item (80\%) Suppose we have a function $f: A \rightarrow B$. Let $f^{*}: \mathscr{P} B \rightarrow \mathscr{P} A$ be the function such that for any $Y \in \mathscr{P} B, f^{*} Y=\{x \in A \mid f x \in Y\}$.
  \begin{enumerate}
    \item Break down $f: A \rightarrow B$, this statement assumes that there is a function(a relation(Cartesian product of two elements from A and from B) that is both serial and functional) from set A to set B 
    \item Break down $f^{*}: \mathscr{P} B \rightarrow \mathscr{P} A$. This statement assumes that there exists a power set of B and a power set of A and assumes that there is a function from the one to the other
    \item Combining the condition (a) and condition (b), the question asks us to prove that with all these condition above, we should have a causation structure that $Y \in \mathscr{P} B, f^{*} Y=\{x \in A \mid f x \in Y\}$
    \item Break down $Y \in \mathscr{P} B, f^{*} Y=\{x \in A \mid f x \in Y\}$, 
    This Disgusting Notation represents that 2 assumptions. 
    The first assumptions assumes that there is a set Y that
    is the element of power set of the set B, this
    $\mathscr{P} B$ is consistent with the above 
    definition. The second assumption(here comes the nasty part) $ f^{*} Y=\{x \in A \mid f x \in Y\}$ assumes that there is a set $\{x \in A \mid f x \in Y\}$, the subset of A (such that) for all elements x in set A,
    x is a memeber of the $\{x \in A \mid f x \in Y\}$,
    (if and only if), the subset of A that contains ALL and ONLY those objects x (for which) $f x \in Y$


    \item Above are my understanding of the question and assumptions, and now I will start my proof. None of the above are my assumptions, thoses are assumptions coming from the question itself.
    \item I will start with the first assumption, that there is a function from set A to set B. This is a function that is both serial and functional. This means that for any two elements from set A, there is only one element from set B that is the image of the function. This is a function that is both serial and functional. This means that for any two elements from set A, there is only one element from set B that is the image of the function. This is a function that is both serial and functional. This means that for any two elements from set A, there is only one element from set B that is the image of the function. This is a function that is both serial and functional. This means that for any two elements from set A, there is only one element from set B that is the image of the function. This is a function that is both serial and functional. This means that for any two elements from set A, there is only one element from set B that is the image of the function. This is a function that is both serial and functional. This means that for any two elements from set A, there is only one element from set B that is the image of the function. This is a function that is both serial and functional. This means that for any two elements from set A, there is only one element from set B that is the image of the function. This is a function that is both serial and functional. This means that for any two elements from set A, there is only one element from set B that is the image of the function. This is a function that is both serial and functional. This means that for any two elements from set A, there is only one element from set B that is the image of the function. This is a function that is both serial and functional. This means that for any two elements from set A, there is only one element from set B that is the image of the function. This is a function that is both serial and functional. This means that for any two elements from set A, there is only one element from set B that is the image of the function. This is a function that is both serial and functional. This means that for any two elements from set A, there is only one element from set B that is the image of the function. This is a function that is both serial and functional. This means that for any two elements from set A, there is
  \end{enumerate}
\end{enumerate}
(a) (50\%) Show that $f$ is injective iff $f^{*}$ is surjective.

(b) (30\%) Show that $f$ is surjective iff $f^{*}$ is injective.

\begin{enumerate}
  \setcounter{enumi}{2}
  \item (10\%) Using the Axiom of Separation to show that there is no set that contains all sets. (Hint: adapt the reasoning in Russell's Paradox.)

  \item (10\%) Show that for any set $A$, there is no injective function from $\mathscr{P} A$ to $A$.

  \item (10\%) Suppose that $f: A \rightarrow B$ and $g: B \rightarrow A$ are functions such that for any $x \in A, x=g(f x)$, and for any $y \in B, y=f(g y)$. Show that $g=f-1$.

\end{enumerate}

\end{document}