\documentclass[10pt]{article}
\usepackage[utf8]{inputenc}
\usepackage[T1]{fontenc}
\usepackage{amsmath}
\usepackage{amsfonts}
\usepackage{amssymb}
\usepackage{mhchem}
\usepackage{stmaryrd}
\usepackage{mathrsfs}

\title{Advanced Logic: Problem Set 02 }

\author{}
\date{}


\begin{document}
\maketitle
\begin{itemize}
  \item Due date: Friday, September $16 .$

  \item Your proofs may appeal to any facts stated in the first three lectures.

\end{itemize}
\begin{enumerate}
  \item (80\%) Suppose we have a function $f: A \rightarrow B$. Let $f^{*}: \mathscr{P} B \rightarrow \mathscr{P} A$ be the function such that for any $Y \in \mathscr{P} B, f^{*} Y=\{x \in A \mid f x \in Y\}$.
  \begin{enumerate}
    \item Break down $f: A \rightarrow B$, this statement assumes that there is a function(a relation(Cartesian product of two elements from A and from B) that is both serial and functional) from set A to set B 
    \item Break down $f^{*}: \mathscr{P} B \rightarrow \mathscr{P} A$. This statement assumes that there exists a power set of B and a power set of A and assumes that there is a function from the one to the other
    \item Combining the condition (a) and condition (b), the question asks us to prove that with all these condition above, we should have a causation structure that $Y \in \mathscr{P} B, f^{*} Y=\{x \in A \mid f x \in Y\}$, this is a single direction conditional statement. 
    \item Break down $Y \in \mathscr{P} B, f^{*} Y=\{x \in A \mid f x \in Y\}$, 
    This Disgusting Notation represents that 2 assumptions. 
    The first assumptions assumes that there is a set Y that
    is the element of power set of the set B, this
    $\mathscr{P} B$ is consistent with the above 
    definition. The second assumption states that function f takes in a set called Y and spits out another set that is $\{x \in A \mid f x \in Y\}$ this structure itself is a biconditional structure, it states that assumes that there is a set if condition x is an element of A, then function f can take this element and the result it spits out is an element of set Y. On the other direction, if there exists(we assumes) an element that can be take in by the function f and the result is an element of set Y, then this element also belongs to the set A.


    \item Above are my understanding of the question and assumptions, and now I will start my proof. None of the above are my assumptions, thoses are assumptions coming from the question itself.
    \item Now, I assumes that there exists an arbitrart element $a$ in set A, namely any element(along) in set A but not an $\emptyset$. Let's start the journey of this little $a$. By the definition of power set, any arbitary elment in a set is also in its power set. Thus, $a$ is also in $\mathscr{P} A$. Now, let's consider the result of function of f taking in $a$ and what it's gonna spits out. $f(a)$ would be on set B by (a). We can use another label $a_{1}$, to represents the result of $a$ after going  
     through function f. Since $a_{1}$ is an element of set B, it is also an element of power set of B. Thus, $a_{1}$ is any arbitary element of power set of B. Therefore, $a_{1}$ is an element of power set of B. 
    \item By observing the structure of (d), we found out that there is a label for any arbitary element in power set of B which is Y. Thus, we find out $a_{1}$ and Y are the same. Y is just another name for $a_{1}$. Therefore we prove the first assumption of (d). Now, we should using $a_{1}$ to prove the second assumption.
    \item The second assumption has two structure one is the equation sign and the other is the biconditional set builder.Let's consider the set builder first. 
    \item (left to right) assumes that there is an arbitary element in set A and it is called $x$(left). From the assuption and (a) and conclusion(g), we can conclude that the result of input x into function f which is fx is an element of set Y. Thus, one direction is proved. 
    \item (right to left) assumes that there exists an element x and it will be putted into function, and the result of it is an element of set Y. Since x is an arbitary element and fx is on B, thus x is also an element of set A because of (a). Thus, the second direction is proved.
    \item Combing (i) and (j), we showed the structure of the nasty set. And It is same for the left side of the equation sign. Here is why. 
    \item Since Y is an input variable that we plug in to the function which is $f^{*}Y$, the result of that function is in  $\mathscr{P} A$. So, the journey of $a$ ends with power set of A by jumping from a to $a_{1}$ to Y to $\mathscr{P} A$. Thus, we proved the equation sign.
    \item Similary, that is same from the other direction of the equation sign. As we consider biconditional structure of the set builder which is the result of synthesizing (a) and (b) and (d) and either left($\{x \in A$\}) or right($\{f x \in Y\}$ ) of the set builder. To be clear, in case of (d), we treat fx as a element of set Y and element of power set of B. Thus, x also end its journey at power set of A. Thus, we proved the equation sign from the left side and the right side.
    \item 
  \end{enumerate}
\end{enumerate}
(a) (50\%) Show that $f$ is injective iff $f^{*}$ is surjective.

(b) (30\%) Show that $f$ is surjective iff $f^{*}$ is injective.

\begin{enumerate}
  \setcounter{enumi}{2}
  \item (10\%) Using the Axiom of Separation to show that there is no set that contains all sets. (Hint: adapt the reasoning in Russell's Paradox.)

  \item (10\%) Show that for any set $A$, there is no injective function from $\mathscr{P} A$ to $A$.

  \item (10\%) Suppose that $f: A \rightarrow B$ and $g: B \rightarrow A$ are functions such that for any $x \in A, x=g(f x)$, and for any $y \in B, y=f(g y)$. Show that $g=f-1$.

\end{enumerate}

\end{document}