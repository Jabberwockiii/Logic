\documentclass[10pt]{article}
\usepackage[utf8]{inputenc}
\usepackage[T1]{fontenc}
\usepackage{amsmath}
\usepackage{amsfonts}
\usepackage{amssymb}
\usepackage{mhchem}
\usepackage{stmaryrd}
\usepackage{hyperref}
\hypersetup{colorlinks=true, linkcolor=blue, filecolor=magenta, urlcolor=cyan,}
\urlstyle{same}

\title{Problem Set 3 }


\author{Advanced Logic}
\date{}


\begin{document}
\maketitle

21st September 2022

\begin{enumerate}
  \item (40\%) Suppose that $A$ is a set; $R$ is an injective relation on $A$; and $B$ is a subset of $A$ such that whenever $x \in B$, it is not the case that $R x x$. Let $C$ be the closure of $B$ under $R$. Prove that whenever $x \in C$, it is not the case that $R x x$.
  \begin{enumerate}
    \item To use induction, we need to have 2 steps:
    \item first, show that every element of B does not have property Rxx
    \item second, !Rxx is preserved.
    \item Since B is that whenever x that is an element of B, x is also an element of A. Thus, we have if $x \in B$, $x \in A$, as B is a subset of A, and it is not the case that Rxx
    \item Thus, every member of B has this property
    \item Since C is the closure of B under R, it is that case the property of B is preserved. 
    \item Thus, the inductive steps worked. 
\end{enumerate}
\end{enumerate}
Hint: this will be a proof by induction.

\begin{enumerate}
  \setcounter{enumi}{2}
  \item Prove that the following hold for all $m, n \in \mathbb{N}$\\
(a) $(20 \%)$\\
\text { double } n =n+n\\
Base case: double 0  = 0 (Dz)\\
induction step: \\
Suppose double n = n + n, Then\\
double(suc(n)) = suc(n) + suc(n), Since \\
suc(n) + suc(n) = suc(suc(n+n)), and \\
double(suc(n)) = suc(suc(double n)). (Ds)\\
Therefore, 
double(n) = n + n


(b) $(10 \%)$\\
m+n =n+m\\
Base case: 0 + 0 = 0 + 0\\
Base case can be proved by using (Az) and (As),\\
assume that m is suc(n), then n + suc(n) = suc(n) + n\\
as n + suc(n) = suc(n + n). \\
Since n is just 0 it is proved. \\
Induction Step: \\
Suppose that m + n = n + m \\
by(As), n + suc(m) = suc(n+m)\\
Since n + 0 = n, \\
suc(n+m) + 0 = suc(n + m)
substitute n with 0, then\\
0 + suc(m) = suc(0+m)
suc(m) = suc(m), \\
Thus, m + n = n + m\\
(c) $(10 \%)$\\
$\text { double } n =2 \times n$ \\
base case:
double(0) = 2 $\times 0$
By (Mz), Since $n \times 0 = 0$, 2 = suc(suc(0)). Thus, \\
2 is a case of n.\\
So, (Mz) can be applied to the base case \\
Induction Step:\\
As I showed that n is just any suc(0).\\
As, suc is a property that holds for all natural numbers \\
and (Ms), $n \times suc(m) = (n\times m) + n$\\
and (Ds), $double(suc(n)) = suc(suc(double n))$
since 2 is just suc(suc(0))\\
substitues 2 we can get double n = $2 \times n$ 

(d) $(5 \%)$\\
(e) $(5 \%)$\\
(f) $(5 \%)$\\
$(g)(5 \%)$
\end{enumerate}
$$
\begin{aligned}
0 \times n &=0 \\
(\text { suc } n) \times m &=(n \times m)+m \\
n \times m &=m \times n
\end{aligned}
$$
If double $m=$ double $n$, then $m=n$

Note: in these proofs you may assume the Axiom of Numbers and the following principles about addition, multiplication, and the 'double' function (which follow from their standard recursive definitions):

(Dz)

(Ds)

$(\mathrm{Az})$

(As)

$(\mathrm{Mz})$

(Ms)
$$
\begin{aligned}
\text { double } 0 &=0 \\
\text { double }(\operatorname{suc} n) &=\operatorname{suc}(\operatorname{suc}(\text { double } n)) \\
n+0 &=n \\
n+\operatorname{suc} m &=\operatorname{suc}(n+m) \\
n \times 0 &=0 \\
n \times \operatorname{suc} m &=(n \times m)+n
\end{aligned}
$$
for any $n, m \in \mathbb{N}$. You may also assume the facts I give proofs of in the supplementary document called 'Proofs by induction: a guide'. Also, some of the later proofs will require on earlier ones. Remember too that ' 2 ' abbreviates 'suc(suc 0$)^{\prime}$ '.

\begin{enumerate}
  \setcounter{enumi}{4}
  \item (10\% extra credit) Open the Lean Natural Numbers Game at \href{https://www}{https://www} . ma.imperial. ac. uk/ buzzard/xena/natural\_number\_game/ and play through (at least) the levels 'Tutorial World', 'Addition World', and 'Multiplication World'. To show that you've completed the levels, send us a screenshot of the last level of Multiplication World open on your computer screen, with your name showing somewhere in the screenshot (e.g. in a text editor window).
\end{enumerate}
Note that the facts you're proving in these levels of the game overlap a lot with the ones you're asked to prove in problem 2. So, you might find it helpful to play the game first and tackle problem 2 afterwards. Whichever order you do it in, it should be instructive to look at your proofs in problem 2 with your solutions to the game, and see how they have the same mathematical content.


\end{document}