\documentclass[10pt]{article}
\usepackage[utf8]{inputenc}
\usepackage[T1]{fontenc}
\usepackage{amsmath}
\usepackage{amsfonts}
\usepackage{amssymb}
\usepackage{mhchem}
\usepackage{stmaryrd}

\title{Problem Set 6 }

\author{}
\date{}


\begin{document}
\maketitle
Advanced Logic

9th October 2022

In the following problems, Arith stands for the ' 'signature of the language of arithmetic": recall that this has one individual constant 0 , one singulary function symbol suc, two binary function symbols $+$ and $x$, and one binary predicate $\leq$. See below for a reminder of the relevant definitions of $F V$ and $[t / v]$.

Note: problems 2-4 ask you to prove something about all terms and formulae of an arbitrary first-order signature $\Sigma$. You may, if you wish, just prove these claims for the special case where $\Sigma$ is Arith. This will lengthen the proofs a bit, but you may find it helpful if you're getting tripped up by notation like ' $f\left(t_{1}, \ldots, t_{n}\right)$ ' and ' $R\left(t_{1}, \ldots, t_{n}\right)$ '.

\begin{enumerate}
  \item (a) (30\%) Prove that no term of the language of arithmetic (i.e., no member of Terms(Arith)) contains the character a.
  \item Answer a
\end{enumerate}
(b) $(30 \%)$ Using the result of part (a), prove that no formula of the language of arithmetic (i.e., no member of $\mathcal{L}($ Arith )) contains the character a.

\begin{enumerate}
  \setcounter{enumi}{2}
  \item (a) (15\%) Prove that whenever $s, t \in \operatorname{Terms}(\Sigma)$ and $v \neq F V(s), s[t / v]=s$.
\end{enumerate}
(b) $(10 \%)$ Using the result of part (a), prove that whenever $P \in \mathcal{L}(\Sigma), t \in \operatorname{Terms}(\Sigma)$, and $v \neq F V(P), P[t / v]=P$

\begin{enumerate}
  \setcounter{enumi}{3}
  \item (a) (10\%) Prove that whenever $s, t \in \operatorname{Terms}(\Sigma)$ and $v \in F V(s), F V(s[t / v])=(F V(s) \backslash$ $\{v\}) \cup F V(t)$
\end{enumerate}
(b) (5\%) Using the result of part (a), prove that whenever $P \in \mathcal{L}(\Sigma), t \in \operatorname{Terms}(\Sigma)$, and $v \in F V(P), F V(P[t / v])=(F V(P) \backslash\{v\}) \cup F V(t)$.

Additional problems (10\% extra credit for successfully solving any one of these.)

\begin{enumerate}
  \item Suppose that $v$ and $v^{\prime}$ are distinct variables and $t$ and $t^{\prime}$ are terms in which neither $v$ nor $v^{\prime}$ occur free. Show that (a) for every term $s, s[t / v]\left[t^{\prime} / v^{\prime}\right]=s\left[t^{\prime} / v^{\prime}\right][t / v]$, and (b) for every formula $P, P[t / v]\left[t^{\prime} / v^{\prime}\right]=P\left[t^{\prime} / v^{\prime}\right][t / v]$.

  \item (Optional, ungraded) Define the "standard numeral" function $\langle\cdot\rangle: \mathbb{N} \rightarrow$ Terms(Arith) recursively as follows:

\end{enumerate}
$$
\begin{aligned}
\langle 0\rangle &=0 \\
\langle\operatorname{suc} n\rangle &=\operatorname{suc}(\langle n\rangle)
\end{aligned}
$$
Suppose we have some funtion $g: \operatorname{Var} \rightarrow \mathbb{N}$. Recursively define a function $d$ : Terms $($ Arith $) \rightarrow \mathbb{N}$ such that $d(v)=g(v)$ for every variable $v$ and $d(\langle n\rangle)=n$ for every $n \in \mathbb{N}$. Prove that the function you defined meets these requirements.

\begin{enumerate}
  \setcounter{enumi}{3}
  \item (Optional, ungraded) Suppose that we have a first-order signature $\Sigma$ that is "well-behaved' in the following sense: no function symbol or predicate is an initial substring of any other funtion symbol or predicate, and no function symbol or predicate has a variable as an initial substring. Let PolishTerms $(\Sigma)$ (the set of "terms in Polish notation" over $\Sigma$ ) be the closure of Var under the family of functions $\left\langle s_{1}, \ldots, s_{n}\right\rangle \mapsto f \oplus t_{1} \oplus \cdots \oplus t_{n}$ where $f$ is an $n$-ary function symbol of $\Sigma$. (Note that unlike our usual definition of term, this one does not use parentheses and commas.)
\end{enumerate}
Prove unique readability for Polish terms: i.e. that whenever $s_{1}, \ldots, s_{m}, t_{1}, \ldots, t_{n} \in$ $\operatorname{PolishTerms}(\Sigma), f$ is an $m$-ary function symbol of $\Sigma$, and $g$ is an $n$-ary function symbol of $\Sigma$, if $f s_{1} \ldots s_{m}=g t_{1} \ldots t_{n}$, then $f=g$ and $m=n$ and $s_{1}=t_{1}$ and $\ldots$ and $s_{m}=t_{m}$.

Definitions The notations ' $F V^{\prime}$ and ' $[t / v]$ ' are used ambiguously for two different functions, one defined on $\operatorname{Terms}(\Sigma)$ and the other on $\mathcal{L}(\Sigma)$. (Or we could think of them as standing for the union of those two functions.)

Where $\Sigma$ is a first-order signature with predicates $R_{\Sigma}$, function symbols $F_{\Sigma}$, and arity function $a_{\Sigma}$, the function $F V: \operatorname{Terms}(\Sigma) \rightarrow \mathcal{P}$ (Var) is defined recursively by:
$$
\begin{array}{rlr}
F V(v) & =\{v\} & \text { (for } v \in \operatorname{Var}) \\
F V(c) & =\emptyset & \left.\quad \text { (for } c \in F_{\Sigma} \text { with } a_{\Sigma}(c)=0\right) \\
F V\left(f\left(t_{1}, \ldots, t_{n}\right)\right) & =F V\left(t_{1}\right) \cup \cdots \cup F V\left(t_{n}\right) \quad\left(\text { for } f \in F_{\Sigma} \text { with } a_{\Sigma}(f)=n>0\right)
\end{array}
$$
The companion function $F V: \mathcal{L}(\Sigma) \rightarrow \mathcal{P}(\operatorname{Var})$ is defined recursively as follows:
$$
\begin{aligned}
F V\left(F\left(t_{1}, \ldots, t_{n}\right)\right) &=F V\left(t_{1}\right) \cup \cdots \cup F V\left(t_{n}\right) \quad\left(\text { for } F \in R_{\Sigma} \text { with } a_{\Sigma}(F)=n>0\right) \\
F V\left(t_{1}=t_{2}\right) &=F V\left(t_{1}\right) \cup F V\left(t_{2}\right) \\
F V(\neg P) &=F V(P) \\
F V(P \wedge Q) &=F V(P) \cup F V(Q) \\
F V(P \vee Q) &=F V(P) \cup F V(Q) \\
F V(P \rightarrow Q) &=F V(P) \cup F V(Q) \\
F V(\forall v P) &=F V(P) \backslash\{v\} \\
F V(\exists v P) &=F V(P) \backslash\{v\}
\end{aligned}
$$
For any $t \in \operatorname{Terms}(\Sigma)$ and $v \in \operatorname{Var}$, the function $[t / v]: \operatorname{Terms}(\Sigma) \rightarrow \operatorname{Terms}(\Sigma)$ (written in postfix position) is defined recursively by
$$
\begin{array}{rlr}
u[t / v] & =\left\{\begin{array}{ll}
t & \text { if } u=v \\
u & \text { if } u \neq v
\end{array} \quad \quad \text { (for } u \in \operatorname{Var}\right)
\end{array}
$$
The companion function $[t / v]: \mathcal{L}(\Sigma) \rightarrow \operatorname{Terms}(\Sigma)$ is defined recursively as follows:
$$
\begin{aligned}
F\left(t_{1}, \ldots, t_{n}\right)[t / v] &=F\left(t_{1}[t / v], \ldots, t_{n}[t / v]\right) \\
(\neg P)[t / v] &=\neg(P[t / v]) \\
(P \wedge Q)[t / v] &=P[t / v] \wedge Q[t / v] \\
(P \vee Q)[t / v] &=P[t / v] \vee Q[t / v] \\
(P \rightarrow Q)[t / v] &=P[t / v] \rightarrow Q[t / v] \\
(\forall u P)[t / v] &= \begin{cases}\forall u P & \text { if } v=u \\
\forall u(P[t / v]) & \text { if } v \neq u \text { and }(u \notin F V(t) \text { or } v \notin F V(P)) \\
\text { undefined } & \text { if } v \neq u \text { and } u \in F V(t) \text { and } v \in F V(P)\end{cases} \\
\{\exists(\exists u P)[t / v]&= \begin{cases}\exists u \\
\exists u(P[t / v]) & \text { if } v \neq u \text { and }(u \notin F V(t) \text { or } v \notin F V(P)) \\
u n d e f i n e d & \text { if } v \neq u \text { and } u \in F V(t) \text { and } v \in F V(P)\end{cases}
\end{aligned}
$$


\end{document}