\documentclass[10pt]{article}
\usepackage[utf8]{inputenc}
\usepackage[T1]{fontenc}
\usepackage{amsmath}
\usepackage{amsfonts}
\usepackage{amssymb}
\usepackage{mhchem}
\usepackage{stmaryrd}
\usepackage{hyperref}
\hypersetup{colorlinks=true, linkcolor=blue, filecolor=magenta, urlcolor=cyan,}
\urlstyle{same}

\title{Problem Set 4 }


\author{Advanced Logic}
\date{}


\begin{document}
\maketitle

26th September 2022

Throughout this problem set, $A$ is some arbitrary set, and $A^{*}$ is the set of lists over $A ; s, t, u$ are arbitrary members of $A^{*}$. $\oplus$ denotes the list concatenation operation on $A^{*}$, defined to obey the following recursion clauses:
$$
\begin{aligned}
{[] \oplus t } &=t \\
(a: s) \oplus t &=a:(s \oplus t)
\end{aligned}
$$
Note that you can and should rely on earlier results in proving later ones.

\begin{enumerate}
  \item $(15 \%)$ Prove that if $s \oplus t=s$ then $t=[]$.
    \begin{enumerate}
        \item $s \oplus t = s$
        \item iff $(s:[]) \oplus t = s$, since $[s] = (s:[])$ 
        \item iff $s:([]\oplus t) = s$, by clause 2
        \item iff $s : t = s$, by clause 1
        \item iff $t = []$
        \item therefore, we reach our conclusion here.
    \end{enumerate}
  \item (15\%) Prove that if $s \oplus t=[]$ then $s=t=[]$.
    \begin{enumerate}
        \item by the injective property of Axiom of the List, we have that $[]$ is not in the range of any $cons_{a}$
        \item $if s \oplus t = []$, then $s$ and $t$ are all $[]$
        \item Thus, s = t = []
    \end{enumerate}

  \item $(15 \%)$ Prove that if $s \oplus t=(a: u)$ then either $s=[]$ or $s=\left(a: s^{\prime}\right)$ for some $s^{\prime}$.
  \begin{enumerate}
    \item Assume that a is just any arbitrary element in A, then $(a : s')$ is just the transformation of list s after taking out an element in A. 
    \item iff $s \oplus t = (a:s') \oplus t = a :(s' \oplus t)$, by clause 2
    \item the case of $s = []$ is tritvial because a : t and a : u is the same as both u and t are all arbitrary elements 
    \item Moreover, for the second case, $s'$ and t and u are all arbitrary, thus, it is the same. 

  \end{enumerate}

\end{enumerate}
For the following problems, we define a function final : $A^{*} \rightarrow \mathcal{P}\left(A^{*}\right)$ recursively as follows:
$$
\begin{aligned}
\text { final }[] &=\{[]\} \\
\operatorname{final}(a: s) &=\text { final } s \cup\{(a: s)\}
\end{aligned}
$$
We say that $s$ is a final sublist of $t$ iff $s \in$ final $t$.

\begin{enumerate}
  \setcounter{enumi}{4}
  \item $(15 \%)$ Prove that [] is a final sublist of every list.
  \begin{enumerate}
    \item Assume that X is the set that has all the list and $x \in X$, then for every final function, we have $x = a: s$ and $\operatorname{final}(a: s) =\text { final } s \cup\{(a: s)\}$, since x is monotonicly popping out elements, it will reach [] as always. 
    \item Thus [] will always be in the form of $a : s$ and then we can use clause 1 to get $\{[]\}$. Thus, we have the conclusion
  \end{enumerate}
  \item $(10 \%)$ Prove that every list is a final sublist of itself.
  \begin{enumerate}
    \item Assume that x is an arbitrary list of all list, then $\operatorname{final}(a: s) =\text { final } s \cup\{(a: s)\}$ by clause 2, and x is (a:s), thus, we have ${final} (s) \cup \{x\}$. Since x $\in final x$, we say that x is a final sublist of itself. Therefore, as x is arbitrary, every list is a final sublist of itself. 
  \end{enumerate}
  \item $(10 \%)$ Prove that $t$ is a final sublist of $s \oplus t$.
  \begin{enumerate}
    \item t $\in final(s \oplus t)$ iff ${final}(s \oplus t)\cup \{t\}$ 
    \item $s \oplus t = (s:[]) \oplus t $
  \end{enumerate}

  \item $(10 \%)$ Prove that if $s$ is a final sublist of $t$, then $t=u \oplus s$ for some $u$.

  \item $(10 \%)$ Prove that if $s \oplus s^{\prime}=t \oplus t^{\prime}$ then either $s^{\prime}$ is a final sublist of $t^{\prime}$ or $t^{\prime}$ is a final sublist of $s^{\prime}$.

  \item (10\% extra credit) Play through the level 'Advanced Multiplication World' in the Lean Natural Numbers Game (\href{https://www}{https://www} . ma. imperial . ac . uk/ buzzard/xena/natural\_number\_game/). To show that you've completed the levels, send us a screenshot of the last level of Advanced Multiplication World open on your computer screen, with your name showing somewhere in the screenshot (e.g. in a text editor window).

\end{enumerate}

\end{document}